\documentclass{article}
\usepackage[utf8]{inputenc}
\usepackage{amsmath,amssymb,amsthm}
\usepackage{nicefrac}
\usepackage{bm}
\usepackage{hyperref}
\usepackage{xurl}
\usepackage[nottoc]{tocbibind}

\newtheorem*{lemma}{Lemma}

\title{Zano $\nicefrac{d}{v}$-CLSAG review}
\author{Cypher Stack\thanks{\url{https://cypherstack.com}}}
\date{\today}

\begin{document}

\maketitle

This report describes the findings of a review of the $\nicefrac{d}{v}$-CLSAG preprint\footnote{\url{https://github.com/hyle-team/docs/tree/master/zano/dv-CLSAG-extension}} by Zano, using the initial version of the preprint at commit \href{https://github.com/hyle-team/docs/tree/182043dd497dab8dc66464028b791641e4f7a56e/zano/dv-CLSAG-extension}{\texttt{182043d}}.
It reflects a limited time scope and represents a best effort; as with any review, it cannot guarantee correctness or security, or that the preprint is otherwise free of errors.
Further, it cannot guarantee that any particular implementation of the $\nicefrac{d}{v}$-CLSAG construction is correct, secure, or suitable for intended use cases.

The author asserts no warranty and disclaims liability for its use.
The author further expresses no endorsement of any kind.
This report has not undergone any further formal or peer review.

The author gratefully thanks Brandon Goodell for helpful collaboration.

\tableofcontents


\section{Executive summary}

The preprint describes $\nicefrac{d}{v}$-CLSAG, a modification to the concise linkable spontaneous anonymous group (CLSAG) signature construction \cite{clsag} that generalizes the generators that may be used.
The CLSAG design was originally written for use in the Monero transaction protocol.
The generalization presented in the $\nicefrac{d}{v}$-CLSAG preprint is intended to support multiple-asset transactions in Zano transaction protocols, but may have other use cases.

The general technique used in the construction is straightforward, in that components of verification keys may use different generators.
These generators may be reused between components, and need not have any particular structure or independence.

Because the $\nicefrac{d}{v}$-CLSAG design follows that of CLSAG closely, the preprint is structured cleanly to directly show relevant modifications.
The preprint uses the same numbering system as the CLSAG preprint, but does not reproduce sections or results that are unchanged.
This makes it very easy to follow the intent of the design and the structure of security proofs.

We find several trivial or minor issues, mainly dealing with notation and clarity.
Many of these are common to the CLSAG preprint as well, but are provided so as to provide a more complete review.

We find several major issues, some of which are common to CLSAG.

One issue relates to a requirement that ensures verification cannot short-circuit hash function evaluations.
This report provides a straightforward fix that adds a check to signature verification.\footnote{The implementation of CLSAG in the Monero codebase already included this check.}

The last two issues apply to the proof of unforgeability.
One of these relates to the definition of a $\kappa$-one-more discrete logarithm ($\kappa$-OMDL) player that wraps a signature forger in the proof of unforgeability.
The other relates to a mismatch between the corruption oracle queries issued by a signature forger and those used by the $\kappa$-OMDL player in the response to its challenger in the proof of unforgeability, and applies to CLSAG as well.
We provide modifications to the proof that address both issues.

We stress that while the unforgeability findings strictly render the proof invalid, we do not find practical vulnerabilities arising from them.


\section{Findings}

We present findings from the review.
Note that some findings comprise errors and issues, but others include explanatory material that may be useful for inclusion in future revisions to aid the reader.

Findings are organized according to the relevant sections and results from the preprint.

As noted, some of the findings apply to CLSAG as well.
Because CLSAG is deployed as part of the Monero protocol, we reached out to project researchers to inform them of the relevant findings in advance of this report.

Subsequent to an initial draft of this report, the Zano team updated the $\nicefrac{d}{v}$-CLSAG preprint in commit \href{https://github.com/hyle-team/docs/tree/eeb53967b1f11e0178df834d7d3e3e6121ff20ff/zano/dv-CLSAG-extension}{\texttt{eeb5396}} to address findings as they saw fit.
While we refer the reader to the updated preprint for details on any changes, we briefly indicate them throughout this section for completeness.


\subsection{Section 3: \texorpdfstring{$\nicefrac{d}{v}$}{d/v}-CLSAG construction}


\subsubsection{Definition 10: minor findings}

\begin{itemize}
	\item The $\mathsf{Setup}$ algorithm states that the generators $G_0, \ldots, G_{v-1}$ are sampled uniformly at random.
	This seems to contradict the statement in Section 1 indicating that the generators may have an efficiently-computable nontrivial discrete logarithm relation.
	While this property is not directly required in the analysis, the inconsistency should be resolved.

	\item It is implicit in the definition that the generators $G_0, \ldots, G_{v-1}$ be nonzero.
	However, this fact is important for later analysis, and we suspect it may be easy for implementers to overlook it.
	As a result, it may be useful to explicitly state this requirement in the definition.

	\item The $\mathsf{Setup}$ algorithm uses the symbol $\bm{G}$ to refer to the vector of generators used for key components.
	However, the symbol is not defined as part of Definition 10 until the $\mathsf{KeyGen}$ algorithm.
	It should be defined before use, or defer use of the symbol until it is defined later.

	\item The $\mathsf{KeyGen}$ algorithm is missing set brackets: it uses the notation ${z_j}_{j=1}^{d-1}$ instead of $\{ z_j \}_{j=1}^{d-1}$ (the brackets must be properly escaped in order to render).

	\item The $\mathsf{Sign}$ algorithm requires that $\bm{pk}_l = \bm{sk} \circ (G_{g(0)}, \ldots, G_{g(d-1)})$.
	This notation can be simplified to $\bm{pk}_l = \bm{sk} \circ \bm{G}$.

	\item The $\mathsf{Sign}$ algorithm defines the values $\{ \mathfrak{D}_j \}, W_{k,i}, \mathfrak{W}_k, w_k$ using indexes $i, j, k$.
	While the bounds on these indexes may be inferred from context, it would be useful to list them directly to avoid confusion.

	\item The $\mathsf{Sign}$ algorithm defines $L_{k,l}, R_{k,l}, c_{l+1}$ and each $L_{k,i}, R_{k,i}, C_{i+1}$ differently than in CLSAG; these changes should be highlighted for consistency, and ideally should have the $k$ index bound listed directly.
	
	\item The $\mathsf{Verify}$ algorithm misspells ``or'' as ``of''.
	
	\item The $\mathsf{Verify}$ algorithm fails if $Q \not\subset \texttt{G}^{d \times n}$ for some $n$.
	This check should require that $Q \subset \texttt{G}^{d \times n}$ using the previously-defined $n$; it should not be arbitrary.
	
	\item The $\mathsf{Verify}$ algorithm fails if $\sigma \not\in \texttt{F}_p^{n'v' + 1} \times \texttt{G}^d$ for some $n', v'$.
	It also does so if $n' \neq n$ or $v' \neq v$.
	These checks can be simplified to the single check that $\sigma \in \texttt{F}_p^{nv + 1} \times \texttt{G}^d$.

	\item Similarly to the $\mathsf{Sign}$ algorithm, the $\mathsf{Verify}$ algorithm should list the bounds on indexes $i, j, k$ when defining values that use them.
	
	\item The $\mathsf{Link}$ algorithm, in its full-key-oriented linkability variant, uses $\mathfrak{W}$ and $\mathfrak{W}'$ without defining them.
	They should be parsed from the signatures $\sigma$ and $\sigma'$, respectively.
	The given linking tags parsed from the signatures are not used.

	\item The $\mathsf{Link}$ algorithm, in its single-key-oriented linkability variant, parses auxiliary group elements $\{ \mathfrak{D}_j \}_j$ and $\{ \mathfrak{D}'_j \}_j$ but does not use them.
\end{itemize}

\textbf{Action}: The Zano team addressed each of these issues in the preprint, with the exception of index bound clarification.
They indicated a desire not to deviate from the original CLSAG preprint except where necessary, and emphasized that the bounds could be inferred from context and careful reading.


\subsection{Section 4: proofs of security}


\subsubsection{Lemma 2: major finding}

The first major finding identified in Theorem 2 requires an addition to the lemma.
We give such a modification here, noting that it does not address the additional minor finding.

\begin{lemma}
	For values $Q, \mathfrak{T}, \{\mathfrak{D}_j\}_j$ as defined in Definition 10, let $\{\mu_j\}_j$ also be defined as in Definition 10.

	\begin{enumerate}
		\item For all $k$ and for all $i$, the mapping $$(Q, \mathfrak{T}, \{\mathfrak{D}_j\}_j) \mapsto \sum_{j:g(j)=k} \mu_j Z_{i,j}$$ is collision resistant.

		\item For all $k$, the mapping $$(Q, \mathfrak{T}, \{\mathfrak{D}_j\}_j) \mapsto \sum_{j:g(j)=k} \mu_j \mathfrak{D}_j$$ is collision resistant.
	\end{enumerate}
\end{lemma}

\textbf{Action}: The Zano team added this modified lemma to the preprint.


\subsubsection{Lemma 2: minor findings}

\begin{itemize}
	\item The statement of the lemma conditions on $Q \subset \mathcal{PK}$, but then further conditions on a signing key $sk \in \mathcal{PK}$, which is a contradiction.
	This can be fixed by simply indicating that $sk$ must correspond to a verification key in $Q$.
\end{itemize}

\textbf{Action}: This is effectively superseded by the corresponding major finding, so no action was needed.


\subsubsection{Theorem 2: major finding 1}

The proof states that if the adversary \texttt{A} outputs a valid signature, all verification challenges must be computed by an actual oracle query and must be well ordered.
This reasoning is identical to that in the corresponding CLSAG proof, and in both cases the linkable spontaneous anonymous group (LSAG) signature preprint \cite{lsag} is referenced for a proof of the claim.

The LSAG design is similar to that of CLSAG, though it uses a different security model.
In the preprint, the claim is discussed as part of the proof of its Theorem 1 in Appendix B.
While the details of its reasoning are not essential here, the proof assumes a correspondence between verifier random oracle queries and those made by a forging adversary that retains the relative ordering.

This intuitively makes sense, as each verification query in LSAG, CLSAG, and $\nicefrac{d}{v}$-CLSAG effectively embeds the output of the previous query.
However, it is crucial to observe that this embedding is done multiplicatively.

Using the $\nicefrac{d}{v}$-CLSAG verifier notation, this embedding is done in each $L_{k,i}$ and $R_{k,i}$.
Specifically, $$L_{k,i} = s_{k,i} G_k + c'_i W_{k,i}$$ embeds $c'_i$ multiplicatively against $W_{k,i}$, and $$R_{k,i} = s_{k,i} H_i + c'_i \mathfrak{W}_k$$ embeds $c'_i$ multiplicatively against $\mathfrak{W}_k$.

The issue arises if $W_{k,i} = \mathfrak{W}_k = 0$ for all $k$ and any $i$.
If this is the case, the verifier's computation of $c'_{i+1}$ no longer requires $c'_i$.
As the verifier's only check is that $c'_n = c_0$, it no longer needs to obtain all $c'_i$ through random oracle queries.

This means that the reasoning of the LSAG proof (and, therefore, those of CLSAG and $\nicefrac{d}{v}$-CLSAG) effectively requires that each purported verifier query ensure a proper embedding of the previous query through at least one such nonzero multiplication per query.

There are multiple ways to accomplish this, using the particular definitions of each relevant multiplicand.
We recall the definitions of the multiplicands in question:
\begin{eqnarray*}
	W_{k,i} &=& \sum_{j:g(j)=k} \mu_j Z_{i,j} \\
	\mathfrak{W}_k &=& \sum_{j:g(j)=k} \mu_j \mathfrak{D}_j
\end{eqnarray*}

It suffices to show that at least one of the following holds:
\begin{enumerate}
	\item There exists $k$ such that $\mathfrak{W}_k \neq 0$.
	\item For each $i$, there exists $k$ (possibly distinct across $i$) such that $W_{k,i} \neq 0$.
\end{enumerate}
We claim that it further suffices for the verifier to check that $\mathfrak{T} \neq 0$, which is computationally efficient.
To see why, observe that if all $\{\mathfrak{D}_j\}_j = 0$, then $\mathfrak{W}_k = 0$ for all $k$.
The definition of each $\mathfrak{W}_k$ is precisely one of the mappings given in the modified Lemma 2; since the mapping is collision resistant, it is second-preimage resistant and therefore setting $\mathfrak{T} \neq 0$ will not map to zero except with negligible probability.

It is also possible to check that any other $\{\mathfrak{D}_j\} \neq 0$, or (using the other mapping in the modified Lemma 2 with the same reasoning) that there exists a nonzero key component $Z_{i,j}$ for all $i$; however, these are either less efficient or of equivalent efficiency.

To summarize, this issue may be addressed by updating the $\mathsf{Verify}$ algorithm in Definition 10 to check that $\mathfrak{T} \neq 0$.

\textbf{Action}: The Zano team added this check to the preprint.


\subsubsection{Theorem 2: major finding 2}

The proof describes an algorithm \texttt{M} that plays the $\kappa$-OMDL game introduced in Definition 1, and which wraps the forking algorithm $\mathcal{F}^B$, ultimately working down to the adversary \texttt{A}.

The definition of \texttt{M} presented in the proof is flawed at several points.

It states that \texttt{M} receives a set $S = \{ \widetilde{pk}_i \}_{i=0}^{q-1}$ of discrete logarithm public keys from its challenger.
However, this is poorly defined; the $\kappa$-OMDL game described in Definition 1 fixes a single group generator $G \in \texttt{G}$ that it uses to produce its challenge points.
This generator is not specified in the proof.

After receiving the challenge points, \texttt{M} next partitions them into vectors of $d$ keys each, suitable for presentation to the forking algorithm:
\begin{alignat*}{2}
	pk_0 &= (Z_{0,0}, \ldots, Z_{0,d-1}) &&= (\widetilde{pk}_0, \ldots, \widetilde{pk}_{d-1}) \\
	pk_1 &= (Z_{1,0}, \ldots, Z_{1,d-1}) &&= (\widetilde{pk}_d, \ldots, \widetilde{pk}_{2d-1}) \\
	&&\vdots
\end{alignat*}
In the case where the forking algorithm $\mathcal{F}^B$ returns a pair of valid signature triples representing a successful forgery, the structure of the public keys means that the forgery is only valid when the public key generator vector $\bm{G}$ consists of only the generator $G$ used by the $\kappa$-OMDL challenger.
The more general case of arbitrary generators intended for $\nicefrac{d}{v}$-CLSAG in Definition 10, where different components of each $pk_i$ may use distinct generators from $\bm{G}$, cannot apply.
Specifically, the generator $G_k$ given in the discusssion of the success probability of \texttt{M} is only known after the forgery is produced; however, the $\kappa$-OMDL game requires it to be specified in advance.

We further note ambiguity in the computation of weighting coefficients $\{\mu_j\}_j$ returned from a successful forgery by $\mathcal{F}^B$.
Recall that the forking algorithm returns two valid signature tuples: $(m, Q, \sigma)$ and $(m, Q, \sigma')$.
Because each coefficient $\mu_j$ uses the linking tag auxiliary group elements from a signature as random oracle input, it is unclear which signature ($\sigma$ or $\sigma'$) \texttt{M} should use to compute the set $\{\mu_j\}_j$.
Fortunately, it does not matter which is used; however, this should be specified.

This observation also leads to a further ambiguity.
The definition states that if all weighting coefficients are zero, \texttt{M} effectively fails.
It should be specified whether this occurs if \textit{either} set of weighting coefficients (those from $\sigma$ or from $\sigma'$) is zero, or if \textit{both} sets are zero.
In either case, this likelihood is negligible.

Further, the algorithm can be simplified in this case by not having \texttt{M} sample random values to return to its challenger, but simply to return failure.
Indeed, the random sampling specified in the definition returns the set $\{w_k\}_k$ as one component of its response to the challenger, but the challenger expects only a single value for this component.

\textbf{Action}: This is effectively superseded by the next major finding, so no action was needed.


\subsubsection{Theorem 2: major finding 3}

In the proof, the algorithm \texttt{M} plays the $\kappa$-OMDL game with public keys supplied by its challenger by wrapping the forking algorithm $\mathcal{F}^B$ to produce a pair of valid forged signatures.

The $\kappa$-OMDL game given in Definition 1 requires that the number of challenger keys $\kappa + 1$ used in the player's response exceed the number of corruption oracle queries the player makes; that is, the player may corrupt only up to $\kappa$ keys.

The statement of the proof states that if there exists a forger making $\kappa'$ corruption oracle queries, then the algorithm \texttt{M} is a solver of the $2d\kappa'$-OMDL game.
This bound purportedly arises because each of the $\kappa'$ corruption oracle queries made by the forger corresponds to a public key with $d$ components, each of which is corrupted by \texttt{M}.
Further, the use of the forking algorithm $\mathcal{F}^B$ to produce a pair of forged signatures produces the factor of $2$ in the bound: each of the forged signatures involves $d\kappa'$ corrupted public key components.

However, the response returned by \texttt{M} to its challenger fails to meet the required oracle query bounds specified in Definition 1.
Because \texttt{M} makes up to $2d\kappa'$ corruption oracle queries to its challenger, the game definition requires that it respond with at least $2d\kappa' + 1$ keys used in its discrete logarithm relationship.
While the number of keys returned is not directly specified in the proof, it can be inferred from context that \texttt{M} returns only $d$ keys, corresponding to those returned by the forger.

This means \texttt{M} does not win the $2d\kappa'$-OMDL game.

It's possible to fix this issue by modifying both the underlying hardness assumption and the definition of \texttt{M}.

Specifically, we claim that we can replace the $\kappa$-OMDL game with a standard discrete logarithm game.
In this game, the player \texttt{M} receives a single public key generated by its challenger with respect to a fixed group generator, and must respond with the corresponding discrete logarithm.
We require that the discrete logarithm challenger set up its game such that the generator $G$ corresponds to the $j = 0$ index of $\nicefrac{d}{v}$-CLSAG public keys.

This means that \texttt{M} must interact differently with the forking algorithm forger $\mathcal{F}^B$.
The modified algorithm does the following:
\begin{itemize}
	\item Receives a public key $X$ from its challenger.
	\item Samples $\{z_{i,j}\}_{i,j=0}^{q-1,d-1} \in \texttt{F}$.
	\item Sets up $\nicefrac{d}{v}$-CLSAG public keys $$pk_i = (z_{i,0}, \ldots, z_{i,d-1}) \circ \bm{G} = (Z_{i,0}, \ldots, Z_{i,d-1})$$ and then sets $Z_{i',0} = X$ for a randomly-sampled index $i'$.
	\item Plays the $\nicefrac{d}{v}$-CLSAG game with the forking adversary using the set of public keys.
	\item On receiving a corruption oracle query, \texttt{M} responds with the corresponding discrete logarithms if the query does not correspond to $pk_{i'}$, and fails if the query corresponds to $pk_{i'}$.
	\item If the forking algorithm fails, or if it succeeds on a forking index $l$ not corresponding to the returned public key set to $pk_{i'}$, \texttt{M} fails.
	\item Otherwise, \texttt{M} uses the pair of forged signatures to compute the discrete logarithm of $X$, which it returns to its challenger.
\end{itemize}

Note that this changes the failure conditions on \texttt{M}.
We observe that the index $i'$ is sampled uniformly at random in advance, and is unknown to the forking algorithm and forger; further, the keys produced by \texttt{M} are sampled identically to that of its challenger.
In particular, this means that the adversary cannot base its queries or responses on this index.

The likelihood that the forger does not corrupt the challenge key $pk_{i'}$ in its at most $\kappa'$ corruption oracle queries is described by a hypergeometric distribution, and is bounded by $(q-\kappa')/q$.
The forking algorithm $\mathcal{F}^B$ runs the forger twice, so the likelihood that neither forger queries on this key is bounded by $((q-\kappa')/q)^2$.

The likelihood that the forking algorithm $\mathcal{F}^B$ fails is unchanged, so the existing bound $$\epsilon \left( \frac{\epsilon}{q} - \frac{1}{2^\eta} \right)$$ holds as before.

It is no longer necessary that \texttt{M} check the weighting coefficients to ensure they are not all zero.
This is because in the $\kappa$-OMDL game, the player must produce a nontrivial discrete logarithm relation; in the standard discrete logarithm game, this is no longer the case.
Instead, we only require that $\mu_0 \neq 0$, as it must be invertible for \texttt{M} to compute the required discrete logarithm.
The likelihood that $\mu_0 = 0$ is negligible.

Finally, for \texttt{M} to succeed, the public key set returned by the forking algorithm for its forgeries must contain $pk_{i'}$ at the forking point, noting that both forgeries fork at the same point using a common public key set.
This likelihood is therefore bounded by $1/q$.

This means that the overall likelihood that \texttt{M} succeeds is bounded by $$\left( \frac{q-\kappa'}{q} \right)^2 \epsilon \left( \frac{\epsilon}{q} - \frac{1}{2^\eta} \right) \left( \frac{1}{q} \right)$$ and scales as $O(\epsilon^2)$.

\textbf{Action}: The Zano team updated this proof in the preprint.


\subsubsection{Theorem 2: minor findings}

\begin{itemize}
	\item The statement of the theorem includes the phrasing ``unforgeability exists'' instead of ``unforgeability game exists'' by mistake.
	
	\item Throughout the proof, the hash function $\mathcal{H}^s$ is used for round challenges.
	This is undefined; according to Definition 10, it should be $\mathcal{H}_0^s$ instead.

	\item The proof uses the notation $\mathcal{F}^b$ instead of $\mathcal{F}^B$ at one point to refer to the forking algorithm.
	
	\item The proof includes the phrasing ``number of corruptions oracle queries'' instead of ``number of corruption oracle queries'' by mistake.
\end{itemize}

\textbf{Action}: The Zano team addressed these issues in the preprint.


\subsubsection{Theorem 4: minor findings}

\begin{itemize}
	\item In step 5 of the signature simulation, the phrasing ``indexing modulo n'' is used instead of the mathematically-typeset ``indexing modulo $n$'' by mistake.
	
	\item The first sentence of a paragraph fails to capitalize, using the phrasing ``if $b = 1$'' instead of ``If $b = 1$'' by mistake.
\end{itemize}

\textbf{Action}: The Zano team addressed these issues in the preprint.


\bibliographystyle{plain}
\bibliography{main}

\end{document}
